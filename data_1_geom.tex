\chapter{Geometric Parameters}

\section{Wing Basic Geometric Parameters}

Aspect ratio is given by the following formula: \cite{Raymer1992}
\begin{equation}
  A = \frac{b^2}{S}
\end{equation}

Taper ratio is given by the following formula. \cite{Raymer1992}
\begin{equation}
  \lambda = \frac{c_{tip}}{c_{root}}
\end{equation}

\section{Mean Aerodynamic Chord}

For tapper wing mean aerodynamic chord can be calculated using following formula: \cite{Galinski2016}
\begin{equation}
  c_{MAC} = \frac{2}{3} c_{root} \frac{1+\lambda+\lambda^2}{1+\lambda}
\end{equation}

For more complex shapes mean aerodynamic chord is given as follows: [] % TODO: citation
\begin{equation}
  c_{MAC} = 
  \left(
    \int_{-\frac{b}{2}}^{\frac{b}{2}} \left( c \left( y \right) \right)^2 dy
  \right)
  \div
  \left(
    \int_{-\frac{b}{2}}^{\frac{b}{2}} \left( c \left( y \right) \right) dy
  \right)
\end{equation}

\section{Wing Aerodynamic Center}

Position of wing aerodynamic center ${\vec r}_{AC}$ is at 25\% of the mean aerodynamic chord and its lateral coordinate is given by the following formula. \cite{Raymer1992}, \cite{Galinski2016}, \cite{Torenbeek1982}
\begin{equation}
  y_{AC} =
  \frac{ b \left( 1 + 2 \lambda \right) }{ 6 \left( 1 + \lambda \right) }
\end{equation}
