\chapter{Environment}

\section{Atmosphere}

US Standard Atmosphere 1976 is used to calculate air temperature, pressure, density, viscosity and speed of sound depending on altitude.

Mean molecular weight is given as follows:
\begin{equation}
  M_0 = \frac{ \sum_{j} M_j F_j }{ \sum_{j} F_j } = 28.9645
\end{equation}

Temperature is given by the following formula: \cite{NASA-TM-X-74335}
\begin{equation}
  T \left( h \right)
  =
  T_j + \left( \frac{dT}{dh} \right)_j \left( h - h_j \right)
\end{equation}

Pressure is given as follows: \cite{NASA-TM-X-74335}
\begin{align}
  p \left( h \right)
  =
  p_j \left( \frac{T_j}{ T \left( h \right) } \right)
  ^
  { \frac{gM_0}{ R \left( \frac{dT}{dh} \right)_j } }
  &\mathrm{~for~} \left( \frac{dT}{dh} \right)_j \neq 0 \\
    p \left( h \right)
  =
  p_j e^{ \frac{ g M_0 \left( h - h_j \right) }{RT_j} }
  &\mathrm{~for~} \left( \frac{dT}{dh} \right)_j = 0
\end{align}

Density is expressed by the following formula: \cite{NASA-TM-X-74335}
\begin{equation}
  \rho \left( h \right)
  =
  \frac{ p \left( h \right) M_0 }{ RT \left( h \right) }
\end{equation}

Speed of sound is given as follows: \cite{NASA-TM-X-74335}
\begin{equation}
  c_S \left( h \right)
  =
  \sqrt{ \frac{ \gamma RT \left( h \right) }{ M_0 } }
\end{equation}

Dynamic viscosity is given by the formula: \cite{NASA-TM-X-74335}
\begin{equation}
  \mu \left( h \right)
  =
  \frac{ 1.458 \cdot 10^{-6} \sqrt{ \left[ T \left( h \right) \right]^3 } }
  { T \left( h \right) + S }
\end{equation}

Kinetic viscosity is given as follows: \cite{NASA-TM-X-74335}
\begin{equation}
  \nu \left( h \right)
  =
  \frac{ \mu \left( h \right) }{ \rho \left( h \right) }
\end{equation}

\newpage

\vfill

\begin{table}[h!]
  \begin{center}
    \begin{tabular}{ S | S | S | S }
      \toprule
      \textbf{Altitude} & \textbf{Temperature gradient} & \textbf{Temperature} & \textbf{Pressure} \\
      {$h_j$} & {$\left( \cfrac{dT}{dh} \right)_j$} & {$T_j$} & {$p_j$} \\
      {[m]} & {[K/m]} & {[K]} & {[Pa]} \\ \midrule
          0 & -6.5e-3 & 288.15 & 101325.0    \\
      11000 &  0.0    & 216.65 &  22632.0    \\
      20000 &  1.0e-3 & 216.65 &   5474.8    \\
      32000 &  2.8e-3 & 228.65 &    868.01   \\
      47000 &  0.0    & 270.65 &    110.9    \\
      51000 & -2.8e-3 & 270.65 &     66.938  \\
      71000 & -2.0e-3 & 214.65 &      3.9564 \\
      \bottomrule
    \end{tabular}
    \caption{Reference levels \cite{NASA-TM-X-74335} }
  \end{center}
\end{table}

\vfill

\begin{table}[h!]
  \begin{center}
    \begin{tabular}{ l | S | S }
      \toprule
      \textbf{Gas species} & \textbf{Molecular weight} & \textbf{Fractional volume} \\
      {} & {[kg/kmol]} & {[-]} \\ \midrule
      Nitrogen       & 28.0134  & 0.78084     \\
      Oxygen         & 31.9988  & 0.209476    \\
      Argon          & 39.948   & 0.00934     \\
      Carbon Dioxide & 44.00995 & 0.000314    \\
      Neon           & 20.183   & 0.00001818  \\
      Helium         & 4.0026   & 0.00000524  \\
      Krypton        & 83.8     & 0.00000114  \\
      Xenon          & 131.3    & 0.000000087 \\
      Methane        & 16.04303 & 0.000002    \\
      Hydrogen       & 2.01594  & 0.0000005   \\
      \bottomrule
    \end{tabular}
    \caption{Molecular weights and fractional volume composition of S/L dry air \cite{NASA-TM-X-74335} }
  \end{center}
\end{table}

\vfill
