\chapter{Workflow}

\section{Blue Marble Next Generation}

\subsection{Elevation Data}

\texttt{gdal\_translate} utility can be used to rescale Blue Marble Next Generation elevation data from 8-bit normalized onto real elevation value. Output files must have 16-bit color depth.

\begin{codelistbash}
  \lstinputlisting{code/terrain_BMNG_elev.sh}
\end{codelistbash}

\subsection{Land Surface, Ocean Color and Sea Ice}

\texttt{gdal\_translate} utility can be used to create Blue Marble Next Generation Land Surface, Ocean Color and Sea Ice georeferenced TIIF (GeoTIFF) files. 

\begin{codelistbash}
  \lstinputlisting{code/terrain_BMNG_land.sh}
\end{codelistbash}

\section{Landsat 7 ETM+}

LibGrid can be used to convert Landsat 7 ETM+ separate bands to true-color image. \cite{Terrain-OpenTerrain}

GDAL can be used for further processing resulting images to match Virtual Planet Builder data format requirements.

\subsection{Extracting}

Following script can be used to extract Landsat data.

\begin{codelistbash}
  \lstinputlisting{code/terrain_Landsat_extract.sh}
\end{codelistbash}

\subsection{Merging Bands}

\texttt{gridcopy} tool can be used to set the no-data value to 0.

\begin{codelistbash}
  \lstinputlisting{code/terrain_Landsat_merge_1.sh}
\end{codelistbash}

\texttt{merger} can be used to merge separate bands to true-color image.

\begin{codelistbash}
  \lstinputlisting{code/terrain_Landsat_merge_2.sh}
\end{codelistbash}

\subsection{Reprojection}

\texttt{gdalwarp} can be used to reproject resulting true-color image to lat/long quasi-projection.

\begin{codelistbash}
  \lstinputlisting{code/terrain_Landsat_reproject.sh}
\end{codelistbash}

\subsection{Merging Tiles}

\texttt{gdalbuildvrt} can be used to merge resulting images.

\begin{codelistbash}
  \lstinputlisting{code/terrain_Landsat_tiles_merge.sh}
\end{codelistbash}

\texttt{gdal\_translate} can be used to convert merged image to GeoTIFF.

\begin{codelistbash}
  \lstinputlisting{code/terrain_Landsat_convert.sh}
\end{codelistbash}

\section{High Resolution Orthoimagery}

\subsection{Extracting}

Following script can be used to extract data.

\begin{codelistbash}
  \lstinputlisting{code/terrain_HiResOrtho_extract.sh}
\end{codelistbash}

\subsection{Scaling}

\texttt{gdal\_translate} can be used to scale down images.

\begin{codelistbash}
  \lstinputlisting{code/terrain_HiResOrtho_scale.sh}
\end{codelistbash}

\subsection{Merging Tiles}

\texttt{gdalbuildvrt} can be used to merge tiles.

\begin{codelistbash}
  \lstinputlisting{code/terrain_HiResOrtho_merge.sh}
\end{codelistbash}

\texttt{gdal\_translate} can be used to convert merged image to GeoTIFF.

\begin{codelistbash}
  \lstinputlisting{code/terrain_HiResOrtho_convert.sh}
\end{codelistbash}

\subsection{Reprojection}

\texttt{gdalwarp} can be used to reproject merged image to lat/long quasi-projection.

\begin{codelistbash}
  \lstinputlisting{code/terrain_HiResOrtho_reproject.sh}
\end{codelistbash}

\section{Vector Map Level 0}

\subsection{Creating Shapefiles}

Following script can be used to create shapefiles of different VMAP0 layers.

\begin{codelistbash}
  \lstinputlisting{code/terrain_VMAP0_create.sh}
\end{codelistbash}

\subsection{Rasterizing}

\texttt{gdal\_rasterize} can be used to rasterize shapefiles.

\begin{codelistbash}
  \lstinputlisting{code/terrain_VMAP0_rasterize.sh}
\end{codelistbash}

\subsection{Converting}

\texttt{osgconv} can be used to convert shapefiles to other file formats.

\begin{codelistbash}
  \lstinputlisting{code/terrain_VMAP0_convert.sh}
\end{codelistbash}

\section{Building Terrain Database}

Virtual Planet Builder can be used to build terrain database.

\subsection{Generating Basic Terrain}

\begin{codelistbash}
  \lstinputlisting{code/terrain_VPB_generate.sh}
\end{codelistbash}

\subsection{Patching Textures}

\begin{codelistbash}
  \lstinputlisting{code/terrain_VPB_patch_tex.sh}
\end{codelistbash}

\subsection{Patching Elevation}

\begin{codelistbash}
  \lstinputlisting{code/terrain_VPB_patch_dem.sh}
\end{codelistbash}

