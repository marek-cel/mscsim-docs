\chapter{Aerodynamics}

Aerodynamic forces are calculated in Aerodynamic Axis System, while moments are calculated in Stability Axis System. Rotation matrix from Stability Axis System to Body Axis System can be calculated using formula (\ref{eq-aero-matrix-alpha}). Rotation matrix from Aerodynamic Axis System to Body Axis System can be calculated using following formulas:
\begin{equation}
  \label{eq-aero-matrix-alpha}
  \boldsymbol T \left( \alpha \right)
  =
  \left[
    \begin{matrix}
      -1 & 0 &  0 \\
       0 & 1 &  0 \\
       0 & 0 & -1 \\
    \end{matrix}
  \right]
  \left[
    \begin{matrix}
      \cos \alpha & 0 & -\sin \alpha \\
                0 & 1 &            0 \\
      \sin \alpha & 0 &  \cos \alpha \\
    \end{matrix}
  \right]
  =
  \left[
    \begin{matrix}
      -\cos \alpha & 0 &  \sin \alpha \\
                 0 & 1 &            0 \\
      -\sin \alpha & 0 & -\cos \alpha \\
    \end{matrix}
  \right]
\end{equation}

\begin{equation}
  \label{eq-aero-matrix-beta}
  \boldsymbol T \left( \beta \right)
  =
  \left[
    \begin{matrix}
       \cos \beta & \sin \beta & 0 \\
      -\sin \beta & \cos \beta & 0 \\
                0 &          0 & 1 \\
    \end{matrix}
  \right]
\end{equation}

\begin{equation}
  \boldsymbol T \left( \alpha, \beta \right)
  =
  \boldsymbol T \left( \alpha \right) \boldsymbol T \left( \beta \right)
  =
  \left[
    \begin{matrix}
      -\cos \alpha \cos \beta & -\cos \alpha \sin \beta &  \sin \alpha \\
                  -\sin \beta &              \cos \beta &            0 \\
      -\sin \alpha \cos \beta & -\sin \alpha \sin \beta & -\cos \alpha \\
    \end{matrix}
  \right]
\end{equation}

Considering a no-wind conditions angle of attack and angle of sideslip (positive when the aircraft velocity component along the transverse axis is positive \cite{ISO-1151-1-1988}) are given as follows:
\begin{align}
  \alpha &= \arctan \left( \frac{w}{ \sqrt{ u^2 + v^2 } } \right) \\
  \beta  &= \arcsin \left( \frac{v}{V} \right)
\end{align}

\section{Tail-off Aircraft}

Tail-off aircraft aerodynamics model is intended to be used in application, e.g. fixed-wing aircrafts, where asymmetric aerodynamic effects, such as autorotation spin or roll damping, are significant.

Forces and moments are calculated for each half-wing to consider asymmetric effects. Half wing aerodynamic center velocity vector used to calculate angle of attack, angle of sideslip as well as forces and moments is given as follows:
\begin{equation}
  \label{eq-aero-v-ac}
  {\vec V}_{AC} = {\vec V}_O + {\vec \Omega} \times {\vec r}_{AC}
\end{equation}

Forces and moments generated by the half-wing are given as follows: \cite{StevensLewis1992}
\begin{align}
  {\vec F}_a &= \left[ F_{X,a}, F_{Y,a}, F_{Z,a} \right]^T \\
  {\vec M}_s &= \left[ M_{X,s}, M_{Y,s}, M_{Z,s} \right]^T
\end{align}

Where:
\begin{align}
  \label{eq-aero-fxa}
  F_{X,a} &= \frac{1}{2} \rho V^2 S C_D \\
  \label{eq-aero-fya}
  F_{Y,a} &= \frac{1}{2} \rho V^2 S C_Y \\
  \label{eq-aero-fza}
  F_{Z,a} &= \frac{1}{2} \rho V^2 S C_L \\
  M_{X,s} &= \frac{1}{2} \rho V^2 S \hat c C_l \\
  M_{Y,s} &= \frac{1}{2} \rho V^2 S \hat c C_m \\
  M_{Z,s} &= \frac{1}{2} \rho V^2 S \hat c C_n
\end{align}

Forces and moments generated by the half-wing expressed in Body Axis System are given by the following formulas:
\begin{align}
  \label{eq-aero-fb}
  {\vec F}_b &= {\boldsymbol T} \left( \alpha, \beta \right) {\vec F}_a \\
  {\vec M}_b &=
  {\boldsymbol T} \left( \alpha \right) {\vec M}_s
  +
  {\vec r}_{AC,b} \times {\vec F}_b
\end{align}

\section{Fuselage}

Fuselage aerodynamics model is intended to be used in application where asymmetric aerodynamic effects can be neglected, e.g. to model helicopter fuselage. It is very much like, described above, tail-off aircraft model. The main difference is that calculations are performed for whole fuselage unlike the tail-off aircraft where calculations are performed for each half-wing.

\section{Stabilizers}

Velocity vector used to calculate stabilizer angle of attack, angle of sideslip as well as forces and moments is calculated using expression (\ref{eq-aero-v-ac}).

Horizontal stabilizer angle of attack is modified due to incidence angle and downwash angle, what can be expressed as follows. \cite{Etkin1972}
\begin{equation}
  \Delta \alpha_h
  =
  i_h + \frac{ \partial \epsilon }{ \partial \alpha } \alpha
\end{equation}

Forces generated by stabilizers are calculated using formulas (\ref{eq-aero-fxa}), (\ref{eq-aero-fya}) and (\ref{eq-aero-fza}).

Formula (\ref{eq-aero-fb}) can be used to calculate stabilizer generated forces expressed in Body Axis System.

It is assumed that horizontal stabilizer generates only drag and lift, while vertical stabilizer generates only drag and side force. Moments generated by stabilizers comes only from force acting on arm, other moments are neglected.
\begin{equation}
  {\vec M}_b = {\vec r}_{AC,b} \times {\vec F}_b
\end{equation}
