\chapter{Flight Dynamics Model}

\section{Assumptions}

Following assumptions are made:
\begin{itemize}
  \item[---] forces and moments acting on the aircraft are considered to be quasi-steady,
  \item[---] aircraft is considered to be a rigid body,
  \item[---] mass and moments of inertia depend only on variable masses (fuel, payload, etc.).
\end{itemize}

\section{Equations of Motion}

\subsection{Dynamic Equations}

Dynamic equations of motion are derived in Body Axis System for a rigid aircraft using conservation of momentum and angular momentum principles which are given by the following formulas. \cite{Taylor2005}, \cite{Osinski1997}, \cite{Leyko2002}
\begin{align}
  \label{eq-fdm-mom-deriv-1}
  \frac{d {\vec P}_b}{dt}
  &=
  \sum_{j} {\vec R}_{j,b} \\
  \label{eq-fdm-ang-mom-deriv-1}
  \frac{d {\vec H}_{O,b}}{dt} + {\vec V}_{O,b} \times {\vec P}_b
  &=
  \sum_{j} {\vec Q}_{O,j,b}
\end{align}

Where:
\begin{align}
  \sum_{j} {\vec R}_{j,b}
  &=
  {\vec R}_{A,b} + {\vec R}_{M,b} + {\vec R}_{LG,b} + {\vec R}_{P,b} \\
  \sum_{j} {\vec Q}_{O,j,b}
  &=
  {\vec Q}_{O,A,b} + {\vec Q}_{O,M,b} + {\vec Q}_{O,LG,b} + {\vec Q}_{O,P,b}
\end{align}

Momentum and angular momentum are: \cite{Osinski1997}, \cite{Leyko2002}
\begin{align}
  \label{eq-fdm-mom-1} 
  {\vec P}_b
  &=
  m {\vec V}_{CM,b} \\
  \label{eq-fdm-ang-mom-1}
  {\vec H}_{O,b}
  &=
  {\boldsymbol I}_{O,b} {\vec \Omega}_b
  +
  m \left( {\vec r}_{CM,b}
  \times
  {\vec V}_{O,b} \right)
\end{align}

Center of mass velocity is:
\begin{equation}
  \label{eq-fdm-vel-cm}
  {\vec V}_{CM,b}
  =
  {\vec V}_{O,b} + {\vec \Omega}_b \times {\vec r}_{CM,b}
\end{equation}

Substituting equation (\ref{eq-fdm-vel-cm}) into equations (\ref{eq-fdm-mom-1}) and (\ref{eq-fdm-ang-mom-1}) gives:
\begin{align}
  \label{eq-fdm-mom-2} 
  {\vec P}_b
  &=
  m {\vec V}_{O,b} + {\vec \Omega}_b
  \times
  {\vec S}_b \\
  \label{eq-fdm-ang-mom-2}
  {\vec H}_{O,b}
  &=
  {\boldsymbol I}_b {\vec \Omega}_b
  +
  {\vec S}_b \times {\vec V}_{O,b}
\end{align}

Where:
\begin{equation}
  {\vec S}_b = \left[ S_X, S_Y, S_Z \right]^T = m {\vec r}_{CM,b}
\end{equation}

Derivatives of momentum and angular momentum in rotating reference frame are: \cite{Taylor2005}, \cite{Osinski1997}, \cite{Leyko2002}
\begin{align}
  \label{eq-fdm-mom-deriv-2} 
  \frac{d {\vec P}_b}{dt}
  &=
  \frac{\delta {\vec P}_b}{\delta t}
  +
  {\vec \Omega}_b \times {\vec P}_b \\
  \label{eq-fdm-ang-mom-deriv-2}
  \frac{d {\vec H}_{O,b}}{dt}
  &=
  \frac{\delta {\vec H}_{O,b}}{\delta t}
  +
  {\vec \Omega}_b \times {\vec H}_{O,b}
\end{align}

Substituting equations (\ref{eq-fdm-mom-deriv-2}) and (\ref{eq-fdm-ang-mom-deriv-2}) into (\ref{eq-fdm-mom-deriv-1}) and (\ref{eq-fdm-ang-mom-deriv-1}) gives:
\begin{align}
  \label{eq-fdm-mom-deriv-3}
  \frac{\delta {\vec P}_b}{\delta t}
  &=
  \sum_{j} {\vec R}_{j,b}
  -
  {\vec \Omega}_b \times {\vec P}_b \\
  \label{eq-fdm-ang-mom-deriv-3}
  \frac{\delta {\vec H}_{O,b}}{\delta t}
  &=
  \sum_{j} {\vec Q}_{O,j,b}
  -
  {\vec V}_{O,b} \times {\vec P}_b
  -
  {\vec \Omega}_b \times {\vec H}_{O,b}
\end{align}

Differentiating equations (\ref{eq-fdm-mom-2}) and (\ref{eq-fdm-ang-mom-2}) gives:
\begin{align}
  \label{eq-fdm-mom-deriv-4}
  \frac{\delta {\vec P}_b}{\delta t}
  &=
  m \frac{\delta {\vec V}_{O,b}}{\delta t}
  +
  \frac{\delta {\vec \Omega}_b}{\delta t}
  \times
  {\vec S}_b \\
  \label{eq-fdm-ang-mom-deriv-4}
  \frac{\delta {\vec H}_{O,b}}{\delta t}
  &=
  {\boldsymbol I}_b \frac{\delta {\vec \Omega}_b}{\delta t}
  +
  {\vec S}_b
  \times
  \frac{\delta {\vec V}_{O,b}}{\delta t}
\end{align}

Substituting equations (\ref{eq-fdm-mom-deriv-4}) and (\ref{eq-fdm-ang-mom-deriv-4}) into (\ref{eq-fdm-mom-deriv-3}) and (\ref{eq-fdm-ang-mom-deriv-3}) gives:
\begin{align}
  \label{eq-fdm-mom-3}
  m \frac{\delta {\vec V}_{O,b}}{\delta t}
  +
  \frac{\delta {\vec \Omega}_b}{\delta t} \times {\vec S}_b
  &=
  \sum_{j} {\vec R}_{j,b}
  -
  {\vec \Omega}_b \times {\vec P}_b \\
  \label{eq-fdm-ang-mom-3}
  {\boldsymbol I}_b \frac{\delta {\vec \Omega}_b}{\delta t}
  +
  {\vec S}_b \times \frac{\delta {\vec V}_{O,b}}{\delta t}
  &=
  \sum_{j} {\vec Q}_{O,j,b}
  -
  {\vec V}_{O,b} \times {\vec P}_b
  -
  {\vec \Omega}_b \times {\vec H}_{O,b}
\end{align}

Representing vector cross product as matrix-vector multiplication equations (\ref{eq-fdm-mom-3}) and (\ref{eq-fdm-ang-mom-3}) can be written as:
\begin{equation}
  \label{eq-fdm-mom-4}
  \left[
    \begin{matrix}
      m & 0 & 0 \\
      0 & m & 0 \\
      0 & 0 & m \\
    \end{matrix}
  \right]
  \left[
    \begin{matrix}
      \dot u \\
      \dot v \\
      \dot w \\
    \end{matrix}
  \right]
  +
  \left[
    \begin{matrix}
         0 &  S_Z & -S_Y \\
      -S_Z &    0 &  S_X \\
       S_Y & -S_X &    0 \\
    \end{matrix}
  \right]
  \left[
    \begin{matrix}
      \dot p \\
      \dot q \\
      \dot r \\
    \end{matrix}
  \right]
  =
  \sum_{j} {\vec R}_{j,b} - {\vec \Omega}_b \times {\vec P}_b
\end{equation}

\begin{multline}
  \label{eq-fdm-ang-mom-4}
    \left[
      \begin{matrix}
        I_X    & -I_{XY} & -I_{XZ} \\
        -I_{XY} &  I_Y    & -I_{YZ} \\
        -I_{XZ} & -I_{YZ} &  I_Z    \\
      \end{matrix}
    \right]
    \left[
      \begin{matrix}
        \dot p \\
        \dot q \\
        \dot r \\
      \end{matrix}
    \right]
    +
    \left[
      \begin{matrix}
          0 & -S_Z &  S_Y \\
        S_Z &    0 & -S_X \\
        -S_Y &  S_X &    0 \\
      \end{matrix}
    \right]
    \left[
      \begin{matrix}
        \dot u \\
        \dot v \\
        \dot w \\
      \end{matrix}
    \right]
    = \\ =
    \sum_j {\vec Q}_{O,j,b}
    -
    {\vec V}_{O,b} \times {\vec P}_b
    -
    {\vec \Omega}_b \times {\vec H}_{O,b}
\end{multline}

Combined equations (\ref{eq-fdm-mom-4}) and (\ref{eq-fdm-ang-mom-4}) can be written as follows. \cite{Sibilski2004}
\begin{equation}
  \label{eq-fdm-motion-1}
  \boldsymbol M \dot {\boldsymbol s} = \boldsymbol R
\end{equation}

Where:
\begin{equation}
  \dot {\boldsymbol s}
  =
  \left[ \dot u, \dot v, \dot w, \dot p, \dot q, \dot r \right]^T
\end{equation}

\begin{equation}
  {\boldsymbol R}
  =
  \left[
    \begin{array}{c}
      \sum_{j} {\vec R}_{j,b} - {\vec \Omega}_b \times {\vec P}_b \\
      \sum_{j} {\vec Q}_{O,j,b} - {\vec V}_{O,b} \times {\vec P}_b - {\vec \Omega}_b \times {\vec H}_{O,b}
    \end{array}
  \right]
\end{equation}

\begin{equation}
 \boldsymbol M
 =
  \left[
    \begin{matrix}
         m &    0 &    0 &    0    &     S_Z &    -S_Y \\
         0 &    m &    0 & -S_Z    &       0 &     S_X \\
         0 &    0 &    m &  S_Y    &    -S_X &       0 \\
         0 & -S_Z &  S_Y &  I_X    & -I_{XY} & -I_{XZ} \\
       S_Z &    0 & -S_X & -I_{XY} &  I_Y    & -I_{YZ} \\
      -S_Y &  S_X &    0 & -I_{XZ} & -I_{YZ} &  I_Z
    \end{matrix}
  \right]
\end{equation}

For the purpose of numerical simulation equation (\ref{eq-fdm-motion-1}) can be written in the following form, which is easy to solve with Gaussian methods.
\begin{equation}
  \label{eq-fdm-motion-2}
  \dot {\boldsymbol s} = {\boldsymbol M}^{-1} \boldsymbol R
\end{equation}

\subsection{Kinematic Equations}

\subsubsection{Time Derivatives}

Position vector derivative is given as follows. \cite{Allerton2009}
\begin{gather}
  %\resizebox{0.8\hsize}{!}{$
  \label{eq-fdm-position-deriv}
  \left[
    \begin{matrix}
      \dot x \\
      \dot y \\
      \dot z
    \end{matrix}
  \right]
  =
    \left[
    \begin{matrix}
       \cos \Theta \cos \Psi &
       \cos \Psi   \sin \Phi   \sin \Theta - \cos \Phi   \sin \Psi &
       \sin \Phi   \sin \Psi + \cos \Phi     \cos \Psi   \sin \Theta \\
       \cos \Theta \sin \Psi &
       \cos \Phi   \cos \Psi + \sin \Phi     \sin \Theta \sin \Psi &
       \cos \Phi   \sin \Theta \sin \Psi -   \cos \Psi   \sin \Phi \\
      -\sin \Theta &
       \cos \Theta \sin \Phi &
       \cos \Phi   \cos \Theta
    \end{matrix}
  \right]
  \left[
    \begin{matrix}
      u \\
      v \\
      w
    \end{matrix}
  \right]
  %$}
\end{gather}

Tait-Bryan angles derivatives are given as follows. \cite{Sibilski2004}, \cite{Allerton2009}
\begin{equation}
  \label{eq-fdm-attitude-deriv}
  \left[
    \begin{matrix}
      \dot \Phi \\
      \dot \Theta \\
      \dot \Psi
    \end{matrix}
  \right]
  =
  \left[
    \begin{matrix}
      1 & \sin \Phi \tan \Theta & \cos \Phi \tan \Theta \\
      0 & \cos \Phi & -\sin \Phi \\
      0 & \sin \Phi \sec \Theta & \cos \Phi \sec \Theta
    \end{matrix}
  \right]
  \left[
    \begin{matrix}
      p \\
      q \\
      r
    \end{matrix}
  \right]
\end{equation}

There are singularities in equation (\ref{eq-fdm-attitude-deriv}) for value of $\Theta$ = $\pm$90\degree . One method of solving this problem is to use quaternions instead of Tait-Bryan angles to describe aircraft attitude.

\subsubsection{Quaternions}

Quaternion time derivative is given as follows. \cite{Sibilski2004}, \cite{StevensLewis1992}
\begin{equation}
  \label{eq-fdm-quaternion-deriv}
  \left[
    \begin{matrix}
      \dot e_0 \\
      \dot e_X \\
      \dot e_Y \\
      \dot e_Z
    \end{matrix}
  \right]
  =
  \left[
    \begin{matrix}
       1 &  p &  q &  r \\
      -p &  0 & -r &  q \\
      -q &  r &  0 & -p \\
      -r & -q &  p &  0
    \end{matrix}
  \right]
  \left[
    \begin{matrix}
      e_0 \\
      e_X \\
      e_Y \\
      e_Z
    \end{matrix}
  \right]
\end{equation}

\section{Numerical Integration}

State vector $\boldsymbol s$ can be calculated by solving initial value problem given by the following expression.
\begin{equation}
  \label{eq-fdm-int-state}
  {\boldsymbol s} \left( t_0 + \Delta t \right)
  =
  {\boldsymbol s} \left( t_0 \right)
  +
  \int_{t_0}^{t_0 + \Delta t} \dot {\boldsymbol s} dt
\end{equation}


State vector derivative $\dot {\boldsymbol s}$ can be calculated using formula (\ref{eq-fdm-motion-2}).

Aircraft position and attitude can be calculated by solving initial value problem given as follows.
\begin{equation}
  \label{eq-fdm-int-position}
  {\boldsymbol x} \left( t_0 + \Delta t \right)
  =
  {\boldsymbol x} \left( t_0 \right)
  +
  \int_{t_0}^{t_0 + \Delta t} \dot {\boldsymbol x} dt
\end{equation}

Where:
\begin{equation}
  {\boldsymbol x}
  =
  \left[
    x, y, z, e_0, e_X, e_Y, e_Z
  \right]^T
\end{equation}

Coordinates vector derivative $\dot {\boldsymbol x}$ can be calculated using formulas (\ref{eq-fdm-position-deriv}) and (\ref{eq-fdm-quaternion-deriv}).

Initial value problems, given by the (\ref{eq-fdm-int-state}) and (\ref{eq-fdm-int-position}) expressions, can be solved using Runge-Kutta 4th-order method which is given as follows. \cite{Press2007}, \cite{Krupowicz1986}, \cite{BaronPiatek2004}
\begin{equation}
  y \left( t_0 + \Delta t \right)
  \approx
  y \left( t_0 \right)
  +
  \frac{1}{6} \Delta t \left( k_1 + 2 k_2 + 2 k_3 + k_4 \right)
\end{equation}

Where:
\begin{align}
  k_1 &=
  f \left( t_n, y_n \right) \\
  k_2 &=
  f \left( t_n + \frac{1}{2} \Delta t, y_n + \frac{1}{2} \Delta t k_1 \right) \\
  k_3 &= 
  f \left( t_n + \frac{1}{2} \Delta t, y_n + \frac{1}{2} \Delta t k_2 \right) \\
  k_4 &= 
  f \left( t_n + \Delta t, y_n + \Delta t k_3 \right) \\
\end{align}
